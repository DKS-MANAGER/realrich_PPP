\documentclass[a4paper,12pt]{article}
\usepackage{graphicx}
\usepackage{amsmath}
\usepackage{float}
\usepackage{booktabs}
\usepackage{geometry}
\geometry{margin=1in}

\title{\textbf{Solution of Second-Order ODE with Boundary Conditions}}
\author{\textbf{Name:} [Your Name] \\ \textbf{Roll Number:} [Your Roll Number]}
\date{\today}

\begin{document}

\maketitle

\section*{Course Information}
\textbf{Course Name:} [Course Name] \\
\textbf{Course Code:} [Course Code]

\section{Problem Statement}
The objective is to solve the following second-order ordinary differential equation (ODE):
\begin{equation}
    \frac{d^2u}{dx^2} = 2, \quad 0 \le x \le 1
\end{equation}
subject to the Dirichlet boundary conditions:
\begin{equation}
    u(0) = 0, \quad u(1) = 0
\end{equation}
This report covers three tasks:
\begin{enumerate}
    \item Derivation of the exact analytical solution.
    \item Numerical solution using the Finite Difference Method (FDM) with $N=5$ and $N=7$ grid points.
    \item Error analysis and comparison of results.
\end{enumerate}

\section{Analytical Solution}
Integrating the ODE $\frac{d^2u}{dx^2} = 2$ with respect to $x$:
\begin{equation}
    \frac{du}{dx} = 2x + C_1
\end{equation}
Integrating again:
\begin{equation}
    u(x) = x^2 + C_1 x + C_2
\end{equation}
Applying the boundary conditions:
\begin{itemize}
    \item At $x=0$, $u(0)=0 \implies 0^2 + C_1(0) + C_2 = 0 \implies C_2 = 0$.
    \item At $x=1$, $u(1)=0 \implies 1^2 + C_1(1) + 0 = 0 \implies C_1 = -1$.
\end{itemize}
Substituting the constants back into the general solution, we obtain the exact analytical solution:
\begin{equation}
    u_{analytical}(x) = x^2 - x
\end{equation}

\section{Numerical Solution Using Finite Difference Method}
\subsection{Methodology}
The domain $[0, 1]$ is discretized into $N$ points with a uniform step size $h = \frac{1}{N-1}$. The second derivative is approximated using the central difference scheme:
\begin{equation}
    \frac{u_{i+1} - 2u_i + u_{i-1}}{h^2} = 2
\end{equation}
Rearranging gives the linear equation for each interior node $i$:
\begin{equation}
    u_{i-1} - 2u_i + u_{i+1} = 2h^2
\end{equation}
This forms a system of linear equations $A \mathbf{u} = \mathbf{b}$, where $A$ is a tridiagonal matrix.

\subsection{Matrix Formulation}
\textbf{Case 1: N = 5} ($h = 0.25$) \\
Number of interior nodes = 3. System size is $3 \times 3$.
\[
\begin{bmatrix}
-2 & 1 & 0 \\
1 & -2 & 1 \\
0 & 1 & -2
\end{bmatrix}
\begin{bmatrix}
u_1 \\ u_2 \\ u_3
\end{bmatrix}
=
\begin{bmatrix}
2h^2 \\ 2h^2 \\ 2h^2
\end{bmatrix}
\]

\textbf{Case 2: N = 7} ($h = 1/6 \approx 0.1667$) \\
Number of interior nodes = 5. System size is $5 \times 5$.
\[
\begin{bmatrix}
-2 & 1 & 0 & 0 & 0 \\
1 & -2 & 1 & 0 & 0 \\
0 & 1 & -2 & 1 & 0 \\
0 & 0 & 1 & -2 & 1 \\
0 & 0 & 0 & 1 & -2
\end{bmatrix}
\begin{bmatrix}
u_1 \\ u_2 \\ u_3 \\ u_4 \\ u_5
\end{bmatrix}
=
\begin{bmatrix}
2h^2 \\ 2h^2 \\ 2h^2 \\ 2h^2 \\ 2h^2
\end{bmatrix}
\]

\subsection{Discrete Solutions}
The numerical solutions obtained for N=5 and N=7 are visualized below.

\begin{figure}[H]
    \centering
    \includegraphics[width=0.8\textwidth]{solution_comparison.png}
    \caption{Comparison of Analytical and FDM Solutions for N=5 and N=7}
    \label{fig:comparison}
\end{figure}

\section{Comparison and Error Analysis}
The pointwise percentage error is defined as:
\begin{equation}
    \text{Error} (\%) = \left| \frac{u_{numerical} - u_{analytical}}{u_{analytical}} \right| \times 100
\end{equation}
(Note: Analysis excludes boundary points where $u=0$).

\subsection{Error Results}
The computed errors were found to be on the order of machine precision ($10^{-16}$).

\begin{table}[H]
    \centering
    \caption{Results for N=5}
    \begin{tabular}{cccc}
        \toprule
        x & $u_{FDM}$ & $u_{Exact}$ & Abs Difference \\
        \midrule
        0.00 & 0.0000 & 0.0000 & 0.0000 \\
        0.25 & -0.1875 & -0.1875 & $\approx 0$ \\
        0.50 & -0.2500 & -0.2500 & $\approx 0$ \\
        0.75 & -0.1875 & -0.1875 & $\approx 0$ \\
        1.00 & 0.0000 & 0.0000 & 0.0000 \\
        \bottomrule
    \end{tabular}
\end{table}

\begin{table}[H]
    \centering
    \caption{Results for N=7}
    \begin{tabular}{cccc}
        \toprule
        x & $u_{FDM}$ & $u_{Exact}$ & Abs Difference \\
        \midrule
        0.00 & 0.0000 & 0.0000 & 0.0000 \\
        0.17 & -0.1389 & -0.1389 & $\approx 0$ \\
        0.33 & -0.2222 & -0.2222 & $\approx 0$ \\
        0.50 & -0.2500 & -0.2500 & $\approx 0$ \\
        0.67 & -0.2222 & -0.2222 & $\approx 0$ \\
        0.83 & -0.1389 & -0.1389 & $\approx 0$ \\
        1.00 & 0.0000 & 0.0000 & 0.0000 \\
        \bottomrule
    \end{tabular}
\end{table}

\section{Discussion}
\begin{itemize}
    \item \textbf{Accuracy:} The FDM solution matches the exact analytical solution almost perfectly (errors $< 10^{-15}$). This is because the central difference scheme is exact for quadratic functions like $u(x) = x^2 - x$. The only errors present are due to floating-point arithmetic.
    \item \textbf{Grid Refinement:} Increasing grid points from 5 to 7 did not significantly change the accuracy in this specific case, as the method had already achieved exactness. For higher-order problems, more points would typically reduce the discretization error.
    \item \textbf{Computational Efficiency:} The tridiagonal system is very efficient to solve (can be solved in $O(N)$ time).
\end{itemize}

\section{Conclusion}
The Finite Difference Method (FDM) was successfully applied to solve $u''=2$. The numerical results showed excellent agreement with the analytical solution $u(x)=x^2-x$. The error analysis confirms that for this specific quadratic problem, the second-order central difference scheme introduces practically zero truncation error.

\end{document}
